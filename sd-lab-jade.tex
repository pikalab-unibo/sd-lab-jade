% !TeX spellcheck = en_GB
\documentclass{beamer}\mode<presentation>{\usetheme{AMSCesenaPurpleAndGold}}
\setbeamertemplate{bibliography item}{\insertbiblabel}

\usepackage{common}

\newcommand{\labN}{2}

\title[L\labN{} -- \jade{} Exercises]{L\labN{} -- \jade{} Exercises}
%
\subtitle[SD]{Distributed Systems / Technologies}
%
\author[Ciatto \and Omicini]
{\emph{Giovanni Ciatto} \and Andrea Omicini\\
\texttt{giovanni.ciatto@unibo.it \and andrea.omicini@unibo.it}}
%
\institute[DISI, Univ. Bologna]
{Dipartimento di Informatica -- Scienza e Ingegneria (DISI)\\\textsc{Alma Mater Studiorum} -- Universit{\`a} di Bologna a Cesena}
%
\date[A.Y. 2020/2021]{Academic Year 2020/2021}

\setbeamercovered{transparent}

\AtBeginSection[]
{
\begin{frame}[c]\frametitle{Next in Line\ldots}
%    \begin{multicols}{2}
        \tableofcontents[sectionstyle=show/shaded, subsectionstyle=show/hide, subsubsectionstyle=hide/hide]
%    \end{multicols}
\end{frame}
}

\begin{document}

\maketitle

\begin{frame}[c]\frametitle{Outline}
    % \begin{multicols}{2}
	    \tableofcontents[sectionstyle=show/show, subsectionstyle=show/show, subsubsectionstyle=show/show]
    % \end{multicols}
\end{frame}

%\section{Motivation}

%\begin{frame}[allowframebreaks]
%\frametitle{Motivation}
%
%    \begin{itemize}
%        \item Some motivation
%
%    \end{itemize}
%
%\end{frame}

% \begin{frame}
% \frametitle{Lecture goals}

%     \begin{itemize}
%         \item some goals
%     \end{itemize}

% \end{frame}

\section{Operations}

\subsection{\jade{} + Gradle}

\begin{frame}{Running \jade{} from Gradle}
    
    We provide two tasks for running \jade{} from Gradle:
    %
    \vfill
    %
    \begin{itemize}
    	\item \alert{\texttt{startPlatform}} --- creates \& starts a new platform with GUI
    	%
    	\begin{itemize}
    		\item default name: \texttt{it.unibo.sd.jade.platform}
    		%
    		\begin{itemize}
				\item (settable through the \texttt{platformName} property)
    		\end{itemize}
    	
    		\item default host: \texttt{localhost}
    		%
    		\begin{itemize}
    			\item (settable through the \texttt{platformHost} property)
    		\end{itemize}
    	
    		\item \alert{no agent} further agent is instantiated
    	\end{itemize}
    
    	\vfill
    	
	    \item \alert{\texttt{startContainer}} --- creates \& starts a new container 
	    %
	    \begin{itemize}
	    	
	    	\item default name: \texttt{it.unibo.sd.jade.container\#\textit{TIMESTAMP}}
	    	%
	    	\begin{itemize}
	    		\item (settable through the \texttt{containerBaseName} property)
	    	\end{itemize}
	    	
	    	\item assumes a platform is running on \texttt{localhost}
	    	%
	    	\begin{itemize}
	    		\item (different hosts can be provided via the \texttt{platformHost} property)
	    	\end{itemize}
    	
    		\item agents to be instantiated can be specified via the \alert{\texttt{agents}} property 
    		%
    		\begin{itemize}
    			\item semi-colon-separated list of string in the form
    			%
    			\begin{center}\ttfamily
    				AGENT\_NAME\alert{:}package.of.AgentClass$\underbrace{\texttt{(arg1, arg2, \ldots)}}_{optional}$-
    			\end{center}
    		\end{itemize}
	    \end{itemize}
    \end{itemize}
\end{frame}

\begin{frame}{General workflow for demos}
	
	\begin{enumerate}
		\item Start a new, empty platform with 
		%
		\begin{itemize}
			\item[\$] \texttt{gradle startPlatform}
		\end{itemize}
		
		\vfill
		
		\item Edit the code of the agent to be discussed (e.g. \texttt{HelloAgent})
		
		\vfill
		
		\item Add the following line to \texttt{gradle.properties}:
		%
		\begin{center}\ttfamily
			\alert{agents}=\textit{helloAgent}:it.unibo.ds.jade.HelloAgent
		\end{center}
	
		\vfill
		
		\item Start a new container containing the \texttt{helloAgent} agent
		%
		\begin{itemize}
			\item[\$] \texttt{gradle startContainer}
		\end{itemize}
	
		\vfill
		
		\item Inspect the logs, draw conclusions
		
		\vfill
		
		\item Close the container $\rightarrow$ Change some code $\rightarrow$ Restart container
		
		\vfill
		
		\item Close the platform	
		
	\end{enumerate}

 \end{frame}

\section{Demos}

\begin{frame}{Agents' code overview}
	\lstinputlisting[language=Java]{./res/src/ExampleAgent.java}
\end{frame}

\startDemo

\subsection{Demo \currentDemo{} -- Atomic actions}

\begin{frame}{Demo \currentDemo{} -- Hello World}
	\lstinputlisting[language=Java]{./res/src/HelloAgent.java}
	%
	\begin{itemize}
		\item what do you expect the agent to do?
	\end{itemize}
\end{frame}

\begin{frame}{Demo \currentDemo{} -- One-Shot Behaviour}

	A behaviour which is executed just once:
	%
	\vfill
	%
	\lstinputlisting[language=Java]{./res/src/OneShotBehaviour.java}
	%
	\begin{itemize}
		\item represents \alert{atomic} actions
	\end{itemize}
\end{frame}

\startDemo

\subsection{Demo \currentDemo{} -- Long-lasting actions}

\begin{frame}{Demo \currentDemo{} -- Hello World\textbf{s}}
	\lstinputlisting[language=Java]{./res/src/HelloAgent2.java}
	%
	\begin{itemize}
		\item what do you expect the agent to do?
	\end{itemize}
\end{frame}

\begin{frame}{Demo \currentDemo{} -- Cyclic Behaviour}
	
	A behaviour which is executed infinitely many times:
	%
	\vfill
	%
	\lstinputlisting[language=Java]{./res/src/CyclicBehaviour.java}
	%
	\begin{itemize}
		\item represents \alert{long-lasting} actions
		\item what's the advantage w.r.t. ordinary \texttt{while(true)} loops?
	\end{itemize}
\end{frame}

\startDemo

\subsection{Demo \currentDemo{} -- Advantages of Internal Scheduling}

%===============================================================================
\section*{}
%===============================================================================
\frame{\titlepage}

%%===============================================================================
%\section*{\bibname}
%%===============================================================================
%
%\setbeamertemplate{page number in head/foot}{}
%%\\\\\\\\\\\\\\\\\\\\\
%\begin{frame}[t,allowframebreaks,noframenumbering]\frametitle{\refname}
%    % \begin{frame}[c]\frametitle{\refname}
%    %	\footnotesize
%    %	\scriptsize
%    \tiny
%    \bibliographystyle{plain}
%    \bibliography{sd-lab-jade}
%\end{frame}
%%\\\\\\\\\\\\\\\\\\\\\

%%%%%%%%%%%%%%%%%%%%%%%%%%%%%%%%%%%%%%%%%%%%%%%%%%%%%%%%%%%%%%%%%%%%%%%%%%%%%%%
\end{document}
%%%%%%%%%%%%%%%%%%%%%%%%%%%%%%%%%%%%%%%%%%%%%%%%%%%%%%%%%%%%%%%%%%%%%%%%%%%%%%%%
